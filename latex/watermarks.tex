\documentclass[a4paper]{llncs}

\usepackage{xcolor}
\usepackage{amsmath} 
\usepackage{hyperref}
\usepackage{graphicx}
\usepackage{subcaption}
\usepackage{booktabs}
\usepackage{siunitx}
\usepackage{pgfplotstable}
\usepackage{tabularx}

\sisetup{
  round-mode          = places, 
  round-precision     = 5, 
}
   
\hypersetup{
    colorlinks=true,
    linkcolor=magenta,
    urlcolor=blue,
    breaklinks,
    citecolor=blue
}

\pgfplotstableset{
  multistyler/.style 2 args={
    @multistyler/.style={display columns/##1/.append style={#2}},
    @multistyler/.list={#1}
  }
}


\newcommand{\bracketR}[1]{\left(#1\right)}
\newcommand{\bracketS}[1]{\left(#1\right)}
\newcommand{\bracketC}[1]{\left(#1\right)}
\newcommand{\innerproduct}[1]{\left<#1\right>}
\newcommand\norm[1]{\left\lVert#1\right\rVert}


\newcommand{\guy}[1]{\marginpar{\textcolor{orange}{Guy: #1}}}
\newcommand{\ben}[1]{\marginpar{\textcolor{blue}{Ben: #1}}}

\begin{document}


\title{Verifying the Resilience of Neural Network Watermarking}

\author{
  Ben Goldberger\inst{1} \and
  Yossi Adi\inst{2} \and
  Joseph Keshet\inst{2} \and
  Guy Katz\inst{1} 
}

\institute{
  The Hebrew University of Jerusalem, Israel \\
  \{jjgold, guykatz\}@cs.huji.ac.il
  \and
Bar Ilan University, Israel \\
\{a, b\}@biu.ac.il
}

\maketitle

\section{Introduction}

% Each of these should be extended into 1-2 paragraphs

Deep Neural Networks (DNN) are a nowadays phenomena in research and in Industry; They're used for large verity of applications, and achieving state of the art result in many fields (Computer vision, speech recognition, AI, and many more). DNNs flexibility and diversity are pushing the limits on what is possible for a computer to solve efficiently. As a result from their empiric success DNNs are now changing the way software is being designed, and broaden the role of Machine Learning trained functions in applications.

Because of the success of DNNs and because designing and training a DNN require certain expertise, time and processing power, the demand for a specifically designed DNN is growing, And Machine Learning as a Service (MLaaS) is now a thing. There are many machine learning services appearing in many forms, from data visualization and cloud computing to frameworks and semi-trained DNNs. This create some issues regarding the rights and ownership of some part of a trained network or design. Because of the relatively simple components of a DNN (Matrices, vectors and simple functions) it's quite easy to copy or use without permission.

In order to deal with such issues we need a way to authenticate a DNN. This may sound simple but the authentication needs to be robust, such that it's hard to remove. Here comes the concept of Digital Watermarking, a way of signing some Digital property such that it's hard to remove the signature, and said signature is unique. There are some proposed method of watermarking a DNN, but it's unclear how effective they are, meaning how difficult it is to remove the watermark from the DNN and what is the effect of removing the watermark.

DNN verification is a new and promising field. We propose a novel
methodology to use verification to measure and verify the robustness
of watermarking techniques.
Main uses of our approach: verify watermarked networks, assess
efficiency of watermarking schemes.

The rest of this paper is organized as follows. In
Section~\ref{sec:background} we provide the necessary background on
DNNs, watermarking, and DNN verification. Next, in
Section~\ref{sec:verifyWatermarks} we introduce our technique for
casting the watermark resilience problem into a verification
problem. Section~\ref{sec:evaluation} describes our implementation and evaluation of the approach on several watermarked DNNs for image
recognition. We discuss related work in Section~\ref{sec:relatedWork},
and conclude in Section~\ref{sec:conclusion}.

\section{Background}
\label{sec:background}

\cite{KaBaDiJuKo17Reluplex,KaHuIbJuLaLiShThWuZeDiKoBa19Marabou}

\section{Defining the minimal change that will change an input decision}
\label{sec:verifyWatermarks}

Given a watermarked trained neural network as described
here \cite{AdBaPiKeWatermarking}.
\guy{make this a proper citation}
\ben{ok}
We
tested what is the minimal change to the network last layer in order
to "remove" some watermarks from the network.


\subsection{Defining the problem for single input}
\label{sec:defineProblem1}

Given a watermarked network $N$ with a set of $K$ watermarks (A set of
inputs to the network) $\bracketsC{x_1,\cdots,x_K}$ we'll mark the network last layer matrix $L$ such that $L$ is a $m\times n$ matrix were $n$
is the layer's number of neurons and $m$ is the network output size.
The change to the last layer will be a matrix with the same dimension
as $L$ we'll mark as $\varepsilon$, such that $\varepsilon_{i,j}$ is the change to the last layer matrix entry $L_{i,j}$. We can describe the network output as  
For a certain input $x$ we're only interested in the input to the last layer we'll mark the input to the last layer $v$. $v$ is a $n\times 1$
vector.  So the original network output $y = Lv$ and the changed
network output is $y' = (L+\varepsilon)v$.
\\\\
For a single input $x$ we denote the original network prediction:
$$
   	d_x := argmax_{i\in \bracketsS{m}}\bracketsC{y_i}
$$
And the changed network prediction:
$$
   	d'_x := argmax_{i\in \bracketsS{m}}\bracketsC{y'_i}
$$
\\
We're interested to find the minimal change to the last layer $\varepsilon$  so that the prediction will change i.e. $d_x\neq d'_x$
\\\\
Well measure the overall change to the layer in two ways
\begin{equation}
\label{eq:normInf}
	\begin{split}
   		\norm{\varepsilon}_{\infty}=max_{i,j}\bracketsC{\abs{\varepsilon_{i,j}}}.
	\end{split}
\end{equation}
And
\begin{equation}
\label{eq:normOne}
	\begin{split}
   		\norm{\varepsilon}_1=\sum_{i,j}\abs{\varepsilon_{i,j}}.
	\end{split}
\end{equation}
\\
For the $\ell_\infty$ norm (\ref{eq:normInf}) with a chosen $d'_x$ that is different from $d_x$ the minimization problem looks like that:
\begin{equation}
\label{eq:LP}
\begin{split}
    Minimize:\quad & M \\
    Subject\ to:\quad & \forall i,j\quad -M \leq\varepsilon_{i,j}\leq M \\
    & y'=(L+\varepsilon)v \\
    & y'_{d_x} \leq y'_{d'_x} \\
	\intertext{Variables are the entries in $\varepsilon,y'$ and $M$}
\end{split}
\end{equation}
\\\\
Similarly for the $\ell_1$ norm (\ref{eq:normOne}) the minimization problem looks like that:
\begin{equation}
\label{eq:NotLP}
\begin{split}
    Minimize:\quad & M \\
    Subject\ to:\quad & \forall i,j\quad -M \leq\sum_{i,j}\abs{\varepsilon_{i,j}}\leq M \\
    & y'=(L+\varepsilon)v \\
    & y'_{d_x} \leq y'_{d'_x} \\
	\intertext{Variables are the entries in $\varepsilon,y'$ and $M$}
\end{split}
\end{equation}
\subsection{Defining the problem for many inputs}
\label{sec:defineProblem2}

Our definition to a single input minimal change $\varepsilon$ to the network last layer $L$ can be extended to more then one input very easily by adding more constraint to the problem.
\\
Given inputs $\bracketsC{x_1,\cdots,x_k}$ and their respective values:
\begin{align*}
\bracketsC{v_1,\cdots,v_k}\quad: & \text{Inputs to the last layer} \\
\bracketsC{d_1,\cdots,d_k}\quad: & \text{Decisions} \\
\intertext{Such that}
\forall 1\leq j\leq k\quad & d_j = argmax_{i\in\bracketsS{m}}\bracketsC{\bracketsR{L v_j}_i}
\intertext{With chosen new desired decisions
$\bracketsC{d'_1,\cdots,d'_k}$ Such that}
\forall 1\leq j\leq k\quad & d'_j \neq d_j
\end{align*}
\\
our minimization problem for the $\ell_\infty$ norm looks like this:
\begin{equation}
\label{eq:LPmany}
\begin{split}
    Minimize:\quad & M \\
    Subject\ to:\quad & \forall i,j\quad -M \leq\varepsilon_{i,j}\leq M\\
    & \forall j\quad y'_j=(L+\varepsilon)v_j \\
    & \forall j\quad \bracketsR{y'_j}_{d_j} \leq \bracketsR{y'_j}_{d'_j}\\
	\intertext{Variables are the entries in $\varepsilon,y'_1,\cdots,y'_k$ and $M$}
\end{split}
\end{equation}
Similarly for the $\ell_1$ norm the minimization problem looks like that:
\begin{equation}
\label{eq:NotLPmany}
\begin{split}
    Minimize:\quad & M \\
    Subject\ to:\quad & \forall i,j\quad -M \leq\sum_{i,j}\abs{\varepsilon_{i,j}}\leq M\\
    & \forall j\quad y'_j=(L+\varepsilon)v_j \\
    & \forall j\quad \bracketsR{y'_j}_{d_j} \leq \bracketsR{y'_j}_{d'_j}\\
	\intertext{Variables are the entries in $\varepsilon,y'_1,\cdots,y'_k$ and $M$}
\end{split}
\end{equation}

\guy{Overall, this looks good. Some stuff will need to be moved to other
sections according to the paper layout.}

\section{Evaluation}
\label{sec:evaluation}
For the $\ell_\infty$ norm when choosing new predictions all the constraint of the minimization problems (\ref{eq:LP}) (\ref{eq:LPmany}) are linear. There for we choose to solve the $\ell_\infty$ problem using the Gurobi linear programming solver.
\\
On the other hand $\ell_1$ norm minimization problems (\ref{eq:NotLP}) (\ref{eq:NotLPmany}) have non linear constraint so a linear programming solver is not enough. Instead we used Marabou which is a verification tool for neural networks that support piecewise-linear activation functions, Marabou can supply a feasible solution to a set of piecewise-linear constraints so combined with the interval halving method we can find the minimal change for the $\ell_1$ case as well. 
\\\\
We tested both approaches on a simple neural network trained on the MNIST data set, the network have one hidden layer with $150$ nodes. The network was watermarked with $100$ images of noise, examples in Figure \ref{fig:noiseExample}.

\begin{figure}
  \centering
  \begin{subfigure}{0.4\linewidth}
    \includegraphics[width=\linewidth]{../data/wm.pdf}
     \caption{Water marks images.}
  \label{fig:noiseExample}
  \end{subfigure}
  \begin{subfigure}{0.4\linewidth}
    \includegraphics[width=\linewidth]{../data/mnist.pdf}
    \caption{MNIST images.}
  \label{fig:mnistExample}
  \end{subfigure}
\caption{Input images examples.}
\label{fig:inputExample}
\end{figure}

\subsection{Removing a single watermark}
The first test we did was to find the minimal change of a single input as defined (\ref{eq:LP}) (\ref{eq:NotLP}) for every water mark (noise) image. For every decision possible beside the original decision we found the minimal change and choose the overall minimum. It turns out that the minimal change was always the second best score in the original prediction output. For example an watermark image $w$ with an original output $y$.
\\
\begin{align*}
d_w=&\underset{i\in\bracketsC{0,\cdots,9}}{argmax}\bracketsC{y_i} \\
\intertext{$d_w$ is the original tagging of $w$.}
d'_w=&\underset{i\in\bracketsC{0,\cdots,9}\setminus\bracketsC{d_w}}{argmax}\bracketsC{y_i} \\
\intertext{$d'_w$ will be the new tagging of $w$.}
\end{align*}
\\
So after we apply change to the last layer of the network the new output $y'$ is such that $\underset{i\in\bracketsC{0,\cdots,9}}{argmax}\bracketsC{y'_i}=d'_w$
\\\\
For each watermark image have found the minimal $\varepsilon$. This can give us some measure of how difficult it is to remove a single watermark image. And there may be some correlation in the minimal change distribution that is not related to the norm as seen in this figure \ref{fig:minSingle}. Beside removing a watermark we're interested in the effect the change we introduced have on the network goal. By applying the change to the original network and evaluating on the MNIST dataset we measured the resulting accuracy of each change. As it turns out we get widely different results from both removal methods (norms) when measuring the accuracy of the changed network. As seen in this table \ref{table:singleWatermark} the $\ell_1$ gives better accuracy result on average but can have a bad accuracy result, compered to the $\ell_\infty$ after removing a single watermark that have a consistent result (there minimal accuracy and the maximal accuracy are quite close) but the average is lower then the $\ell_1$ method.
\\\\

\begin{figure}
  \centering
  \begin{subfigure}{0.4\linewidth}
    \includegraphics[width=\linewidth]{../data/results/problem3/mnist_w_wm_sorted.pdf}
     \caption{The Minimal $\norm{\varepsilon}_{\infty}$ for every watermark image}
  	\label{fig:minSingleLP}
  \end{subfigure}
  \begin{subfigure}{0.4\linewidth}
    \includegraphics[width=\linewidth]{../data/results/problem2/mnist_w_wm_sorted.pdf}
    \caption{The Minimal $\norm{\varepsilon}_1$ for every watermark image}
  	\label{fig:minSingleNotLP}
  \end{subfigure}
  \caption{Notice the scale of the graphs, the $\ell_\infty$ values are much smaller then the $\ell_1$ values. As expected}
\label{fig:minSingle}
\end{figure}
\begin{table}
\resizebox{\linewidth}{!}{
	\pgfplotstabletypeset[
	every head row/.style={before row=\hline,after row=\hline\hline},
	every last row/.style={after row=\hline},
	col sep=comma,
	columns/Norm/.style={string type},
	every column/.style={column type/.add={|}{}},
	every last column/.style={column type/.add={}{|}}
	]{../data/results/mnist_w_wm_1_wm.csv}
	}
\caption{Minimal changes and Accuracy}
\label{table:singleWatermark}
\end{table}

These results begs the question what is the nature of the change and how it differ between the method? Turns out that when minimizing the change according to $\ell_\infty$ (\ref{eq:LP}) the linear programming solver assign all the entries of $\varepsilon$ a small value. This translate to a change to every weight in the last layer which can explain the uniformity of the accuracy test. On th other hand the when minimizing the change according to $\ell_1$ (\ref{eq:NotLP}) the optimal change is to change only a single entry in $\varepsilon$ by a seemingly large value. So only one weight in the last layer is changed. See figure \ref{fig:lastLayerExampleSingle}
\\
 



\begin{figure}
\centering
  \begin{subfigure}{0.4\linewidth}
  \includegraphics[width=\linewidth]{../data/results/problem3/last_layer_1_wm_example.pdf}
     \caption{The Minimal change according to $\ell_\infty$ for a single input}
  \end{subfigure}
  \begin{subfigure}{0.4\linewidth}
    \includegraphics[width=\linewidth]{../data/results/problem2/last_layer_1_wm_example.pdf}
    \caption{The Minimal change according to $\ell_1$ for a single input}
  \end{subfigure}
  \caption{Examples of the change to the last layer. Positive change is colored red and negative change is colored blue.}
  \label{fig:lastLayerExampleSingle}
\end{figure}


\subsection{Removing Multiple watermark}

Removing a single 


\section{Related Work}
\label{sec:relatedWork}

\section{Conclusion and Future Work}
\label{sec:conclusion}


\bibliographystyle{abbrv}
\bibliography{watermarks}

\end{document}

%%% Local Variables:
%%% mode: latex
%%% TeX-master: t
%%% End:
